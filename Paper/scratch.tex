    \draw [->,decorate,decoration={snake, post length=1mm}] (-3.0,0.0)
    -- (-0.1,0.0) node[midway,above] {$\epsilon=h\nu$} ; 
    \draw [->,decorate,decoration={snake, post length=1mm}] (0.1,0.1)
    -- (2.5,1.7) node[midway,sloped,above] {$\epsilon_1=h\nu_1$};
    \draw (1.0,0.0) arc (0:37:1cm);
    \draw (1.2,0.35) node {$\theta$};
    \draw (1.3,-0.7) node {$\mathbf{p},E$};
    \fill (0,0) circle (2pt);
    \draw [->] (0,0) -- (1.5,-1.5);
    \draw (1.75,-1.75) node {$\vec{P}_{ef}$};
    \draw (-3.35,0.0) node {$\vec{P}_{\gamma i}$};
    \draw (2.8,1.85) node {$\vec{P}_{\gamma f}$};



, and imgine further than
the angle between the jet direction and the observer line of sight is
$i$.  

What are the advantages of this method of flagging LISA sources?  Can
you go down to binaries that cannot be resolved with VLBI? Can you go
down in mass or mass ratio or something?


  = 2.42\, \mathrm{degrees}.


%% \begin{figure*}
%% \begin{center}
%%   \begin{tabular}{ccc}
%%     \includegraphics*[width=2.2in]{jet_i40_psi20_beta_0_90.pdf} &
%%     \includegraphics*[width=2.2in]{jet_i40_psi20_beta_0_80.pdf} &
%%     \includegraphics*[width=2.2in]{jet_i40_psi20_beta_0_40.pdf}
%%   \end{tabular}
%% \end{center}
%% \caption{Evolution of the binary separation.  Time $t$ is time since the gravitational wave induced phase began.}
%% \label{fig:srcframe}
%% \end{figure*}

%% \begin{figure*}
%% \begin{center}
%%   \begin{tabular}{ccc}
%%     \includegraphics*[width=2.2in]{jet_i40_psi20_beta_0_90_obs.pdf} &
%%     \includegraphics*[width=2.2in]{jet_i40_psi20_beta_0_80_obs.pdf} &
%%     \includegraphics*[width=2.2in]{jet_i40_psi20_beta_0_40_obs.pdf}
%%   \end{tabular}
%% \end{center}
%% \caption{Evolution of the binary separation.  Time $t$ is time since the gravitational wave induced phase began.}
%% \label{fig:obsframe}
%% \end{figure*}

%% \begin{figure*}
%% \begin{center}
%%   \begin{tabular}{ccc}
%%     \includegraphics*[width=2.2in]{jet_i40_psi20_beta_0_90_ang.pdf} &
%%     \includegraphics*[width=2.2in]{jet_i40_psi20_beta_0_80_ang.pdf} &
%%     \includegraphics*[width=2.2in]{jet_i40_psi20_beta_0_40_ang.pdf}
%%   \end{tabular}
%% \end{center}
%% \caption{Evolution of the binary separation.  Time $t$ is time since the gravitational wave induced phase began.}
%% \label{fig:ang}
%% \end{figure*}

are expected to arise naturally in the context of the
heirarchical formation of galaxies in which galaxies with their black
holes merge and create black hole binaries (cite our paper, illustris
papers, many other papers).  As such binaries are great tools to
understand galaxy formation.  They may also be crucial to explain the
occurancec of massive BHs today, which are routinely invoked for AGN
feedback etc.  Binary BHs also probe the extreme limit of GR.  They
are thus a tool to test GR and probe physics beyond it.

Binary evolution proceeds through three phases.  Describe three
stages.  Out of this binaries spent most of their time on pc scale
separations.  These wide binaries are expected to leave signature in
broad lines regions (cite Shen and Loeb).  There have been many
attempts to detect these wide binaries but none have been detected so
far (cite Decarli).

Aprt from a stochastic gravitational wave background, LISA is expected
to detect GW from individial sources.  These GWs are produced by a
binary BHs. Briefly describe GW chirp here?  If one of these BHs is
accreting, it is expected to have a jet.  Due to orbital motion of the
accreting BH, the jet will precess.  We study this jet precession in
this paper during the last stages of the binary evolution.  Any
peculiar morphological features canact as an electromagnetic signature
of the binary.  

---------------

\begin{figure*}
\begin{center}
  \begin{tabular}{ccc}
    \includegraphics*[width=2.2in]{jet_i40_beta0_90_full.pdf} &
    \includegraphics*[width=2.2in]{jet_i40_beta0_90_zoom1.pdf} &
    \includegraphics*[width=2.2in]{jet_i40_beta0_90_zoom2.pdf}
  \end{tabular}
\end{center}
\caption{Jet morphology with opening angle evolution.}
\label{fig:jet_psi}
\end{figure*}

\begin{figure*}
\begin{center}
  \begin{tabular}{ccc}
    \includegraphics*[width=2.2in]{jet_i40_beta0_90_full.pdf} &
    \includegraphics*[width=2.2in]{jet_i40_beta0_90_zoom1.pdf} &
    \includegraphics*[width=2.2in]{jet_i40_beta0_90_zoom2.pdf}
  \end{tabular}
\end{center}
\caption{Jet morphology with opening angle evolution.}
\label{fig:jet_psi}
\end{figure*}

\begin{figure*}
\begin{center}
  \begin{tabular}{cc}
    \includegraphics*[width=\columnwidth]{{jet_i40_beta0.90_mdot1.0_full}.pdf} &
    \includegraphics*[width=\columnwidth]{{jet_i40_beta0.90_mdot1.0_zoom1}.pdf} 
  \end{tabular}
\end{center}
\caption{$a^4/M^3=1$}
\label{fig:jet_mdot1.0}
\end{figure*}

\begin{figure*}
\begin{center}
  \begin{tabular}{cc}
    \includegraphics*[width=\columnwidth]{{jet_i40_beta0.90_mdot1.00_image_full}.pdf} &
    \includegraphics*[width=\columnwidth]{{jet_i40_beta0.90_mdot1.00_image_zoom1}.pdf} 
  \end{tabular}
\end{center}
\caption{$a^4/M^3=1$}
\label{fig:jet_mdot1.0_image}
\end{figure*}


\begin{figure*}
\begin{center}
  \begin{tabular}{cc}
    \includegraphics*[width=\columnwidth]{{jet_i40_beta0.90_mdot10.00_full}.pdf} &
    \includegraphics*[width=\columnwidth]{{jet_i40_beta0.90_mdot10.00_zoom1}.pdf} 
  \end{tabular}
\end{center}
\caption{$a^4/M^3=10$}
\label{fig:jet_10.0}
\end{figure*}

\begin{figure*}
  \begin{center}
    \begin{tabular}{cc}
      \includegraphics*[width=\columnwidth]{{jet_i40_beta0.90_mdot0.10_full}.pdf} &
      \includegraphics*[width=\columnwidth]{{jet_i40_beta0.90_mdot0.10_zoom1}.pdf} 
    \end{tabular}
  \end{center}
  \caption{$a^4/M^3=0.1$}
  \label{fig:jet_mdot0.1}
\end{figure*}

and
the binary separation shrinks 

For observable predictions we have to project this in observer plane.
Let the angle between the jet direction and the observer line of sight
be $i$.  Then
\begin{equation}
  \psi^\mathrm{obs}\approx\frac{\psi}{\sin i}.
\end{equation}


This gives
\begin{equation}
  \psi^\mathrm{obs}=2.42^\circ \left(\frac{v_\mathrm{jet}}{0.95 c}\right)^{-1}\cos\left(\frac{\chi}{30^{\circ}}\right)\left[\sin\left(\frac{i}{15^{\circ}}\right)\right]^{-1},
\end{equation}
This value in itself is probably not very interesting as it depends on
$v_\mathrm{jet}$ and $\chi$, but it gets interesting when we consider
the evolution of $\psi^\mathrm{obs}$ below.

Binary separation, jet velocity, and the two angles, tell us about the
instantaneous configuration of the jet.  (Recall that we are only
considering an equal mass binary.)  The jet evolution is governed by
the dynamical evolution of the binary separation.  During the
gas-dominated phase, this is given by
\begin{equation}
\frac{a}{\dot a} = \mu/\dot M = 1.1\times 10^7\, \mathrm{yr}\,\dot{\mathcal{M}}^{-1}.   
\end{equation}
This increases by two orders of magnitude during the GW phase to
\begin{equation}
\frac{a}{\dot a} = \frac{5}{256}\frac{c^5a^4}{G^3M^2\mu}=2.53\times 10^5\,\mathrm{yr}\,\frac{a_{16}^4}{M_8^3}.
\end{equation}

Figure shows the effect of this model.

Note that this section introduces a set of geometrical and
astrophysical parameters for the system we started with in the
previous section.  List parameters and mention our assumed values.

\begin{figure*}
\begin{center}
  \begin{tabular}{ccc}
    \includegraphics*[width=2.2in]{a_gw.pdf} &
    \includegraphics*[width=2.2in]{v_orb.pdf} &
    \includegraphics*[width=2.2in]{psi.pdf}
  \end{tabular}
\end{center}
\caption{Evolution of the binary separation in the gravitational wave stage.}
\label{fig:a_gw}
\end{figure*}

\begin{figure*}
\begin{center}
  \begin{tabular}{ccc}
    \includegraphics*[width=2.2in]{a_gas.pdf} &
    \includegraphics*[width=2.2in]{v_orb_gas.pdf} &
    \includegraphics*[width=2.2in]{psi_gas.pdf}
  \end{tabular}
\end{center}
\caption{Evolution of the binary separation in the gas stage.}
\label{fig:a_gas}
\end{figure*}

\begin{figure*}
\begin{center}
  \begin{tabular}{ccc}
    \includegraphics*[width=2.2in]{jet_i40_psi20_beta_0_90_ang.pdf} &
    \includegraphics*[width=2.2in]{jet_i40_psi20_beta_0_80_ang.pdf} &
    \includegraphics*[width=2.2in]{jet_i40_psi20_beta_0_40_ang.pdf}
  \end{tabular}
\end{center}
\caption{Jet structure in angular coordinates. $\beta=0.9, 0.8, 0.4$ from left to right.}
\label{fig:ang}
\end{figure*}



Jet shown in figure has three features: (1) chirp, (2) beam twisting,
and (3) cone opening.  This can be observed using VLBA.  Figure shows
the evolution of the binary when it is in the gas accretion phase.  We
have chosen steady gas accretion with $\dot{\mathcal{M}}=1$ here for
simplicity.  Note that the starting separation here is about 100 pc.
What we find is that in the beginning the binary orbital period is
about $10^7$ yr. So the jet will show no signs of precision at this
point.  However as the binary separation reduces to about 10 pc, the
period reduces to $\sim 100$ yr and the jet will start showing
precision.  At this point the time scale for binary merger is $\sim
10^7$ yr.

In other words \emph{since the gas accretion time scale is of the
  order of the typical jet lifetime} the jet structure will show a
transition from a straight jet to a precessing jet at a separation of
\begin{equation}
  v_\mathrm{apparant}\left(\frac{t_0}{10^7}\right).
\end{equation}
from the binary.  \textbf{[Convert this to arcseconds on the sky.]}
The location at which the jet morphology shows this precision and the
opening angle evolution give a direct constraint on the epoch of
graviational wave emission of the binary.

Thus the new insight here is that the signature of the jet precision
due to GW-inspiral is observationally accessible. 

Variability of lensed quasars constraints the size of accretion disks
to about $10^{15}$ cm ($\sim 10^{-3}$ pc) \citep{2010ApJ...712.1129M}.
This is smaller than the $a_\mathrm{GW}$ by a factor of less than 10.

This means that the GW-induced precision will be visible in the jet
precision for a very short distance near the source before the
accretion disk is disrupted.

Last plots in this note show some example jets.  In all these plots
$\psi$ is the half-opening angle of the conical jet, and $i$ is the
angle between the jet and the line of sight.

A rough argument about observability: For our example, the period is
0.1 yr at $t = 63967.3816055$ yr.  We can assume that the chirp begins
at $t = 10^4$ yr when period is $\sim 2$ yr.  So the total duration of
the chirp is $5.3\times 10^4$ yr.

A 2 yr period gives $\sim 1$ milli-arcsec wiggles in the jet.  So the
idea is that the wiggles will change from milli-arcsec to micro-arcsec
over $\sim 10^5$ yr.  That is over a length scale of $\sim 30$ kpc.

Now the question is: how big an angle would a 30 kpc jet subtend on
the sky in arcsecs?

Let us assume $z = 0.302$ which is the redshift of 1928+738 jet that
Roos (1993) have studied.  At this redshift 30 kpc will subtend about
6.748195445702 arcsec.

So we are talking about mciro- to milli-arcssec scale wiggles in tens
of arcsec scale jets.  Jets are routinely seen over hundreds of
arcsec.

Connect with LISA


From Figures \ref{fig:a_gw} we can make these four points:

\begin{itemize}
\item From the jet structure, we can tell how long until the binary enters GW-induced inspiral.
\item In particular, we can flag binaries that are entering GW-inspiral now.
\item We could locate binaries that are now in GW inspiral phase but
  have helical jet structures that are impossible to resolve.
\item From the above jet structure, we get an idea of the history of gas inflow on the binary.
\end{itemize}

There will also probably be some spectral evolution along the jet.
This spectral evolution will act as a cross check on the evolution
deduced from the jet morphology.  An important advantage of this
technique is that it allows us to locate binaries that may be
impossible to locate even with VLBI.

